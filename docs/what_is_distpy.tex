\section{What is distpy?}

Distributed Acoustic Sensing (DAS) has dramatically changed the way the reservoir and the well is surveyed. 
By providing a distributed array of vibration measurements at up to 10 or 20 kHz on spacings as short as 12 cm, DAS has 
introduced a view of the subsurface that could not previously be achieved. Whether this is the tracking of slugs in a multiphase flow,
 providing very short turnaround times on borehole seismic profiles, or monitoring the slowly evolving geomechanic strain; DAS represents
 a step change in monitoring.
 \paragraph{The challenge of DAS} 
 is the huge data volumes. For example a 3.5km well monitored every 12.75 cm at 10kHz will result in around 410 MB of data every second.
This data needs to be processed to detect events, filtered to extract key signals, the resulting image might go to an image classifier or acoustic-signature detector.
To get started in DAS there is a need for a baseline level of data-handling and scalable parallel processing that enables academic researchers and
DAS customers to explore, for themselves, the possibilities that this data source brings.
 \paragraph{distpy}
is a python a module that automates testing of signal processing workflows and implements generic data sharing between systems.
Signal processing workflows are defined via directed graphs. 
The data is chunked (typically into 1 second windows) and each chunk processed independently, leading to embarassingly parallel processing 
that scales across compute architectures of all shapes and sizes.

